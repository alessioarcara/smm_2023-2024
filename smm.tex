\documentclass{article}

% --- Layout e formattazione generale ---
\usepackage[a4paper, landscape, twocolumn, margin=1.5cm, columnsep=1cm]{geometry}

% --- Matematica ---
\usepackage{amsthm}
\usepackage{amsmath}
\usepackage{amsfonts}
\usepackage{amssymb}
\usepackage{xcolor}
\numberwithin{equation}{section}
\usepackage{mdframed}
\DeclareMathOperator*{\argmax}{arg\,max}
\DeclareMathOperator*{\argmin}{arg\,min}
\usepackage{cancel}

% --- Teoremi, definizioni e ambienti ---
\renewcommand\qedsymbol{$\blacksquare$}
\newtheorem*{definition}{\color{red}\textbf{Definition}}
\newtheorem*{theorem}{\color{green}\textbf{Theorem}}
\newtheorem*{proposition}{\color{blue}\textbf{Proposition}}
\newmdenv[%
    backgroundcolor=lightgray,
    frametitlebackgroundcolor=gray,
    hidealllines=true,
    frametitlefont=\sffamily\bfseries\color{white},
    frametitle={Example},
]{example}

% --- Grafica e disegni ---
\usepackage{graphicx}
\graphicspath{ {./images/} }


\begin{document}
\pagenumbering{gobble}

\tableofcontents
\clearpage

\pagenumbering{arabic}
\parindent 0pt

\section{Linear Algebra Tools}
This chapter introduces inner product to give geometric meaning to vectors and
vector spaces, enabling calculations of length, distance, and angles.

\begin{definition}[Symmetric Positive Definitive Matrix] 
    A symmetric matrix $A\in \mathbb{R}^{n\times n}$ that satisfies
    \begin{equation}\label{positive_definite_matrix}
        \text{for every nonzero vector }x:x^TAx>0
    \end{equation}    
    is called \textbf{positive definite}. If only $\geq$ holds in
    \ref{positive_definite_matrix}, then $A$ is called \textbf{positive
    semidefinite}.
\end{definition}
These properties helps in identifying positive definite matrices without
having to check the definition explicitly:
\begin{enumerate}
    \item The null space of $A$ contains only the null vector; 
    \item The diagonal elements $a_{ii}$ of $A$ are positive; 
    \item The eigenvalues of $A$ are real and positive.
\end{enumerate}
\subsection{Angles and Orthogonality}
The angle $\omega$ between vectors $x$ and $y$ is computed as:
$$\cos\omega=\frac{\langle x,y\rangle}{\lVert x\rVert_2 \lVert y\rVert_2}$$ 
Here, $\langle x,y\rangle$ denotes the inner product between $x$ and $y$.

This angle indicated the vectors' similarity in orientation. 
\begin{definition}[Orthogonal vectors]
    Two vectors are orthogonal if $\langle x,y \rangle=0$. If additionally
    $\lVert x\rVert=1=\lVert y\rVert$, then $x$ and $y$ are orthonormal.
\end{definition}
\begin{definition}[Orthogonal matrix]
    A square matrix is an orthogonal matrix if and only if \underline{its columns are
    orthonormal} so that 
    $$AA^T=I=A^TA$$
    which implies that 
    $$A^{-1}=A^T$$
\end{definition}
The length of a vector $x$ is not changed when transforming it using an
orthogonal matrix $A$.
$$\lVert Ax\rVert_2^2=\lVert x\rVert_2^2$$
Moreover, the angle between any two vectors $x,y$ is also unchanged when
transforming both of them using an orthogonal matrix $A$.
\begin{definition}[Orthonormal Basis]
    In an $n$-dimensional vector space $V$ with a basis set
    $\{b_1,\ldots,b_n\}$, if all the basis vectors are orthogonal to each
    other, the basis is called as an \textbf{orthogonal basis}.
    Additionally, if the length of each basis vector is 1, the basis is
    referred to as an \textbf{orthonormal basis}.
\end{definition}
We can also have vector spaces that are orthogonal to each other. Given a
vector space $V$ of dimension $D$, let's  consider a subspace $U$ of dimension
$M$ such that $U\subseteq V$. Then its \textbf{orthogonal complement}
$U^{\perp}$ is a $D-M$ dimensional subspace $V$ and contains all vectors in
$V$ that are orthogonal to every vector in $U$.
\begin{center}
    \includegraphics[width=0.6\linewidth]{images/orthogonal_complement}
\end{center}
\subsubsection{Orthogonal Projections}
Projections are key linear transformations in machine learning and are
particularly useful for handling high-dimensional data. Often, only a few
dimensions in such data are essential for capturing the most relevant
information. By projecting the original high-dimensional data onto a lower
dimensional feature space, we can work more efficiently to learn about the
dataset and extract significant patterns.
\begin{definition}[Projection]
    Let $V$ be a vector space and $U\subseteq V$ a subspace of $V$. A linear
    mapping $\pi:V\to U$ is called \textbf{projection} if it satisfies
    $\pi^2=\pi\circ\pi=\pi$.
\end{definition}
Given that linear mappings can be represented by transformation matrices, the
above definition extends naturally to \textit{projection matrices} $P_{\pi}$.
These matrices exhibit the property that $P_{\pi}^2=P_{\pi}$.

The projection $\pi_U(x)$ of a vector $x\in \mathbb{R}^n$ onto a subspace $U$
is the closest point necessarily in $U$ to $x$.
\cleardoublepage
\section{Matrix Decompositions}
\subsection{Eigenvalues and Eigenvectors}
Eigenanalysis helps us understand linear transformations represented by a
matrix $A$. Eigenvectors $x$ are special vectors that only get scaled, not
rotated, when multiplied by $A$. The scaling factor is the eigenvalue
$\lambda$, which indicated how much $x$ is stretched or shrunk. $\lambda$ can
also be zero.
\begin{definition}[Eigenvalue and Eigenvector]
    Let $A\in \mathbb{R}^{n\times n}$ be a square matrix. Then $\lambda\in
    \mathbb{R}$ is an \textbf{eigenvalue} of $A$ and nonzero vector $x$ is the
    corresponding \textbf{eigenvector} of $A$ if 
    \begin{equation}\label{eq:eigenvalue_equation}
        Ax=\lambda x
    \end{equation}
    We call \ref{eq:eigenvalue_equation} the \textbf{eigenvalue equation}.
\end{definition}
The following statements are equivalent:
\begin{itemize}
    \item $\lambda$ is an eigenvalue of $A\in \mathbb{R}^{n\times n}$.
    \item A nonzero vector $x$ exists such that $Ax=\lambda x$ or,
        equivalently, $(A-\lambda I_n)x=0$ for $x\neq 0$.
    \item Then $A-\lambda I$ is a \textbf{singular
        matrix} and its determinant is \textbf{zero}.
\end{itemize}
Each eigenvector $x$ has one unique eigenvalue $\lambda$, but each $\lambda$
can have multiple eigenvectors.
\begin{definition}[Eigenspace and Eigenspectrum]
    For $A\in \mathbb{R}^{n\times n}$, the set of all eigenvectors of $A$
    associated with an eigenvalue $\lambda$ spans a subspace of
    $\mathbb{R}^n$, which is called the \textbf{eigenspace} of $A$ with
    respect to $\lambda$ and is denoted by $E_{\lambda}$. The set of all
    eigenvalues of $A$ is called the \textbf{eigenspectrum} of $A$.
\end{definition}
\begin{definition}
    Let $\lambda_i$ be an eigenvalue of a square matrix $A$. Then the
    \textbf{geometric multiplicity} of $\lambda_i$ is the number of linearly
    independent eigenvectors associated with $\lambda_{i}$. In other words, it
    is the dimensionality of the eigenspace spanned by the eigenvectors
    associated with $\lambda_i$.
\end{definition}
\begin{theorem}
    The eigenvectors $x_1,\ldots,x_n$ of a matrix $A\in \mathbb{R}^{n\times
    n}$ with $n$ distinct eigenvalues $\lambda_1,\ldots,\lambda_n$ are
    linearly independent.
\end{theorem}
This theorem states that eigenvectors of a matrix with $n$ distinct eigenvalues
form a basis of $\mathbb{R}^n$.
\cleardoublepage
\section{Vector calculus}
Firstly, we'll explore partial derivatives and gradients, focusing on functions that take a vector as input and produce a
single real number as output. These functions are formally represented as
$f:\mathbb{R}^n\to \mathbb{R}$.

Subsequently, we will extend these ideas to functions that not only take a
vector as input but also produce a vector as output. These functions can be
written as $f:\mathbb{R}^n\to \mathbb{R}^m$.

\subsection{Gradients of Real-Valued Functions}
When we deal with a function that depends on multiple variables, such as 
$f(x)=f(x_1,x_2)$, we use the \textbf{gradient} to represent its derivative.
The gradient is a vector composed of \textbf{partial derivates} of the
function. To compute each partial derivates, we differentiate the function
with respect to one variable while keeping all other variables constant.
\begin{equation}\label{eq:gradient_real_valued_functions}
   \nabla_x f=\begin{bmatrix}
       \frac{\partial{f}}{\partial{x_1}} &
       \frac{\partial{f}}{\partial{x_2}} & \cdots &
       \frac{\partial{f}}{\partial{x_n}}
    \end{bmatrix}\in \mathbb{R}^{1\times n} 
\end{equation}
where $n$ is the number of variables.
\paragraph{Basic Rules of Partial Differentiation}
\begin{itemize}
    \item[] Product rule:
        $$\frac{\partial}{\partial{x}}(f(x)g(x))=\frac{\partial{f}}{\partial{x}}g(x)+f(x)\frac{\partial{g}}{\partial{x}}$$
    \item[] Sum rule:
        $$\frac{\partial}{\partial{x}}(f(x)+g(x))=\frac{\partial{f}}{\partial{x}}+\frac{\partial{g}}{\partial{x}}$$
    \item [] Chain rule:
        $$\frac{\partial}{\partial{x}}(g\circ
        f)(x)=\frac{\partial}{\partial{x}}\left(g(f(x))\right)=\frac{\partial{g}}{\partial{f}}\frac{\partial{f}}{\partial{x}}$$
\end{itemize}

In the context of the chain rule, consider $f$ as implicitly a composition
$f\circ g$.

If a function $f(x_1,x_2)$ is a function of $x_1$ and $x_2$, where
$x_1(t)$ and $x_2(t)$ are themselves functions of a single variable $t$, the
chain rule yields the partial derivates
$$\frac{\text{d}f}{\text{d}t}=\begin{bmatrix}
    \frac{\partial{f}}{\partial{x_1}} & \frac{\partial{f}}{\partial{x_2}}
    \end{bmatrix}\begin{bmatrix}
    \frac{\partial{x_1(t)}}{\partial{t}} \\ 
    \frac{\partial{x_2(t)}}{\partial{t}}
\end{bmatrix}=\frac{\partial{f}}{\partial{x_1}}\frac{\partial{x_1}}{\partial{t}}+\frac{\partial{t}}{\partial{x_2}}\frac{\partial{x_2}}{\partial{t}}$$
\newpage
\begin{example}
    Consider $f(x_1,x_2)=x_1^2+2x_2$, where $x_1=\sin t$ and $x_2=\cos t$,
    then
    $$\text{with}\quad\frac{\partial{f}}{\partial{x_1}}=2x_1,\quad \frac{\partial{f}}{\partial{x_2}}=2$$
    $$\begin{aligned}
        \frac{\text{d}f}{\text{d}t}&=2\sin t \frac{\partial{\sin t}}{\partial{t}}+2 \frac{\partial{\cos
        t}}{\partial t}\\
            &=2\sin t\cos t-2\sin t
    \end{aligned}$$
\end{example}
If a function $f(x_1,x_2)$ is a function of $x_1$ and $x_2$, where $x_1(s,t)$
and $x_2(s,t)$ are themselves functions of two variables $s$ and $t$, the
chain rule yields the partial derivates
$$
\frac{\text{d}f}{\text{d}(s,t)}=\begin{bmatrix}
    \frac{\partial{f}}{\partial{s}} & \frac{\partial{f}}{\partial{t}}
\end{bmatrix} 
$$
where 
$$
\begin{aligned}
    \frac{\partial{f}}{\partial{s}}&=\frac{\partial{f}}{\partial{x_1}}\frac{\partial{x_1}}{\partial{{\color{red}s}}}+\frac{\partial{f}}{\partial{x_2}}\frac{\partial{x_2}}{\partial{{\color{red}s}}}\\
    \frac{\partial{f}}{\partial{t}}&=\frac{\partial{f}}{\partial{x_1}}\frac{\partial{x_1}}{\partial{{\color{blue}t}}}+\frac{\partial{f}}{\partial{x_2}}\frac{\partial{x_2}}{\partial{{\color{blue}t}}}\\
\end{aligned}
$$
Another way to obtain these two partial derivatives is to represent the
previous formula as a row vector containing the partial derivatives of $f$
with respect to $x_1$ and $x_2$. This row vector is then multiplied by a
matrix composed of the partial derivatives of $x_1$ and $x_2$ with respect to
$s$ and $t$. When you perform this multiplication, you get the exact same
result as above.
$$
\begin{bmatrix}
    \frac{\partial{f}}{\partial{s}} & \frac{\partial{f}}{\partial{t}}
\end{bmatrix}=\begin{bmatrix}
    \frac{\partial{f}}{\color{red}\partial{x_1}} &
    \frac{\partial{f}}{\color{blue}\partial{x_2}}
\end{bmatrix}\begin{bmatrix}
    {\color{red}\frac{\partial{x_1}}{\partial{s}}} &
    {\color{red}\frac{\partial{x_1}}{\partial{t}}} \\ 
    {\color{blue}\frac{\partial{x_2}}{\partial{s}}} &
    {\color{blue}\frac{\partial{x_2}}{\partial{t}}}
\end{bmatrix}$$
\newpage
\begin{example}
    Given the following functions:\\
    $g:\mathbb{R}^2\to \mathbb{R}^2\quad g(s,t)=(\sin(t)s, \cos(s)t)$\\ 
    $f:\mathbb{R}^2\to \mathbb{R}\quad f(x_1,x_2)=x_1^2+2x_2$\\ 
    $f\circ g: \mathbb{R}^2\to \mathbb{R}$\\
    Compute $\nabla_{(s,t)}(f\circ g)$ and evaluate $\nabla_{(s,t)}(f\circ g)(0,0)$.
    $$
    \begin{aligned}
        &=\begin{bmatrix}
            2s\sin(t) & 2
        \end{bmatrix}
        \begin{bmatrix}
            \sin(t) & s\cos(t)\\ 
            -t\sin(s) & \cos(s)
        \end{bmatrix} \\ 
        &=\begin{bmatrix}
            2s\sin^2(t)-2t\sin(s) \\ 
            2s^2\sin(t)\cos(t)+2\cos{t}
        \end{bmatrix}=(0,2)
    \end{aligned}
    $$
\end{example}
\subsection{Gradients of Vector-Valued Functions}
We can express a vector-valued function $f:\mathbb{R}^n\to \mathbb{R}^m$ as a
column vector of $m$ real-valued functions $f_i:\mathbb{R}^n\to \mathbb{R}$.
Given an input vector $x=\begin{bmatrix} x_1,\ldots,x_n \end{bmatrix}^T\in
\mathbb{R}^n$, the output is defined as: 
$$
f(x)=\begin{bmatrix} f_1(x)\\
\vdots \\ f_m(x) \end{bmatrix}\in \mathbb{R}^m
$$
\begin{definition}[Jacobian]
    By contrast, in Equation \ref{eq:gradient_real_valued_functions}, each partial
    derivative $\frac{\partial f}{\partial x_i}$ is a column vector.
    $$
    \begin{aligned}
        J=\nabla_x f&=\begin{bmatrix}
            \frac{\partial{f}}{\partial{x_1}} & \cdots &
            \frac{\partial{f}}{\partial{x_n}}
        \end{bmatrix} \\ 
                  &=\begin{bmatrix}
                      \frac{\partial{f_1}}{\partial{x_1}} & \cdots &
                      \frac{\partial{f_1}}{\partial{x_n}} \\ 
                      \vdots & & \vdots \\ 
                      \frac{\partial{f_m}}{\partial{x_1}} & \cdots &
                      \frac{\partial{f_m}}{\partial{x_n}} \\ 
                  \end{bmatrix}\\
            J(i,j)&=\frac{\partial f_i}{\partial{x_j}}
    \end{aligned}
    $$
    The collection of all first-order partial derivatives of a vector-valued
    function $f:\mathbb{R}^n\to \mathbb{R}^m$ is called the \textbf{Jacobian}.
    The Jacobian $J$ is an $m\times n$ matrix.
\end{definition}
\newpage
\begin{example}
    $f:\mathbb{R}^2\to \mathbb{R}^3,\quad f:(x_1,x_2)=\begin{pmatrix}
        x_1+x_2 \\ 
        2x_1^2-x_2 \\ 
        -x_1x_2
    \end{pmatrix}$\\ 
    $J(f):\mathbb{R}^2\to \mathbb{R}^{3\times 2},\quad J(i,j)=\begin{bmatrix}
        1 & 1 \\ 
        4x_1 & -1 \\ 
        -x_2 & -x_1 
    \end{bmatrix},\quad
    J(1,1)=\begin{bmatrix}
        1 & 1 \\ 
        4 & -1 \\ 
        -1 & -1
    \end{bmatrix}$
\end{example}
\begin{example}
   Let us consider the linear model 
   $$y=\Phi\theta$$
   where $\theta\in \mathbb{R}^D$ is a parameter vector, $\Phi\in
   \mathbb{R}^{N\times D}$ are input features, and $y\in \mathbb{R}^N$ are the
   corresponding observations. We define the functions 
   $$\begin{aligned}
       e:\mathbb{R}^D\to \mathbb{R}^N,\quad e(\theta)&=y-\Phi\theta \\
       L:\mathbb{R}^N\to \mathbb{R},\quad L(e)&=\lVert e\rVert_2^2, \quad
       L(\theta)= \lVert y-\Phi\theta\rVert_2^2
   \end{aligned}$$
   This is called a \textbf{least-squares loss} function.\\ We want to find
   $\frac{\partial{L}}{\partial{\theta}}$, which is derivative of the loss
   function with respect to the parameters $\theta$. This will allow us to
   find the optimal $\theta$ that minimizes the loss function $L(\theta)$.

   The chain rule allows us to compute the gradient as 
   $$\frac{\partial{L}}{\partial{e}}={\color{red}\frac{\partial{L}}{\partial{e}}}{\color{blue}\frac{\partial{e}}{\partial{\theta}}}$$
   We know that $\lVert e\rVert_2^2=e^Te$ and so 
   $${\color{red}\frac{\partial{L}}{\partial{e}}=2e^T}\in \mathbb{R}^{1\times
   N}$$
   Furthermore, we obtain 
   $${\color{blue}\frac{\partial{e}}{\partial{\theta}}=-\Phi}\in
   \mathbb{R}^{N\times D}$$
   such that our desired derivative is
   $$
   \nabla{L}_{\theta}=-2e^T\Phi=-2\underbrace{\color{red}(y^T-\theta^T\Phi^T)}_{1\times
   N}\underbrace{\color{blue}\Phi}_{N\times D}\in \mathbb{R}^{1\times D}\\ 
   $$
\end{example}
\subsection{Backpropagation and Automatic Differentiation}
In machine learning, finding optimal model parameters often involves
performing gradient descent. This requires computing the gradient of a
learning objective with respect to the model's parameters. Calculating the
gradient explicitly can be impractical due to the complexity and length of the
resulting derivative equations. To address this, the \textbf{backpropagation} algorithm
was introduced in 1962 as an efficient way to compute these gradients,
particularly for neural networks.



In neural networks, the output $y$ is computed through a multi-layered
function composition $y=(f_K\circ f_{K-1}\circ \cdots f_1)(x)$. Here, $x$ are
the inputs (e.g., images), $y$ are the observations (e.g., class labels).
Each functions $f_i,i=1,\ldots,K$, has its own parameters.
Specifically, in the $i^{th}$ layer, the function is given
$f_i(x_{i-1})=\sigma(A_{i-1}+b_{i-1})$, where $x_{i-1}$ is
the output from layer $i-1$ and $\sigma$ is an activation function. 
\begin{center}
    \includegraphics[width=\linewidth]{nn_forward_pass}
\end{center}
In order to train a neural network, we aim to minimize a loss function $L$
with respect to all parameters $A_j,b_j$ for $j=0,\ldots,K-1$.
Specifically, we're interested in optimizing these parameters to minimize the
squared loss given by
$$L(\theta)=\lVert y-f_K(\theta,x)\rVert^2$$
where $\theta=\{A_0, b_0, \ldots,A_{K-1},b_{K-1}\}$.

To minimize $L(\theta)$ we need to compute its gradients of $L$ to the
parameter set $\theta$. This involes calculating the partial derivatives of
$L$ with respect to the parameters $\theta_j=\left\{A_j,b_j\right\}$ for each
layer $j=0,\ldots,K-1$. The chain rule allows us to determine the partial
derivatives as
$$\begin{aligned}
    \frac{\partial{L}}{\partial{\theta_{K-1}}}&=\frac{\partial{L}}{\partial{f_K}}{\color{blue}\frac{\partial{f_K}}{\partial{\theta_{K-1}}}}
    \\
    \frac{\partial{L}}{\partial{\theta_{K-2}}}&=\frac{\partial{L}}{\partial{f_K}}\boxed{{\color{red}\frac{\partial{f_K}}{\partial{f_{K-1}}}}{\color{blue}\frac{\partial{f_{K-1}}}{\partial{\theta_{K-2}}}}}
    \\
    \frac{\partial{L}}{\partial{\theta_{K-3}}}&=\frac{\partial{L}}{\partial{f_K}}{\color{red}\frac{\partial{f_K}}{\partial{f_{K-1}}}}\boxed{{\color{red}\frac{\partial{f_{K-1}}}{\partial{f_{K-2}}}}{\color{blue}\frac{\partial{f_{K-2}}}{\partial{\theta_{K-3}}}}}
    \\
    \frac{\partial{L}}{\partial{\theta_{i}}}&=\frac{\partial{L}}{\partial{f_K}}{\color{red}\frac{\partial{f_K}}{\partial{f_{K-1}}}\cdots}\boxed{{\color{red}\frac{\partial{f_{i+2}}}{\partial{f_{i+1}}}}{\color{blue}\frac{\partial{f_{i+1}}}{\partial{\theta_{i}}}}}
\end{aligned}$$
The {\color{red}red} terms are partial derivatives of the output of a layer
with respect to its inputs, whereas the {\color{blue}blue} terms are partial
derivatives of the output of a layer with respect to its parameters. 

The key insight of backpropagation is to reuse previously computed derivatives
to avoid redundant calculations. When we've computed the partial derivatives
$\frac{\partial{L}}{\partial{\theta_{i+1}}}$, we can reuse them to efficiently
calculate the partial derivatives
$\frac{\partial{L}}{\partial{\theta}_{i}}$.

It turns out that backpropagation is a special case of a set of techniques
known as \textbf{automatic differentiation}. Automatic differentiation
numerically evaluate the exact (up to machine precision) gradient of a function by working with
intermediate variables and applying the chain rule.

\begin{center}
    \includegraphics[width=\linewidth]{nn_backward_pass}
\end{center}
\begin{example}
    Consider the real-valued function 
    $$f(x)=\sqrt{x^2+\exp(x^2)}+\cos(x^2+\exp(x^2))$$
    Another way to attach this would be to just define some \textit{intermediate
    variables}. Say 
    $$
    \begin{aligned}
        a=x^2\\
        b=\exp(a)\\ 
        c=a+b\\ 
        d=\sqrt{c}\\ 
        e=\cos(c)\\ 
        f=d+e
    \end{aligned}
    $$
    The set of equations that include intermediate variables can be thought of
    as a computational graph 
    \begin{center}
        \includegraphics[width=\linewidth]{nn_backpropagation}
    \end{center}
    By looking at the computation graph, we can compute
    $\frac{\partial{f}}{\partial{x}}$ by working backward from the end of the
    graph and obtain the derivative of each variable, making the use of the
    derivatives of the children of that variable
    $$\begin{aligned}
        \frac{\partial{f}}{\partial{d}}=\frac{\partial{f}}{\partial{e}}=1 \\
        \frac{\partial{f}}{\partial{c}}=\frac{\partial{f}}{\partial{d}}\frac{\partial{d}}{\partial{c}}+\frac{\partial{f}}{\partial{e}}\frac{\partial{e}}{\partial{c}}
        \\ 
        \frac{\partial{f}}{\partial{b}}=\frac{\partial{f}}{\partial{c}}\frac{\partial{c}}{\partial{b}}
        \\
        \frac{\partial{f}}{\partial{a}}=\frac{\partial{f}}{\partial{b}}\frac{\partial{b}}{\partial{a}}+\frac{\partial{f}}{\partial{c}}\frac{\partial{c}}{\partial{a}}
        \\ 
        \frac{\partial{f}}{\partial{x}}=\frac{\partial{f}}{\partial{a}}\frac{\partial{a}}{\partial{x}}
    \end{aligned}$$
    We observe that the computation required for calculating the derivative is
    of similar complexity as the computation of the function itself (forward
    pass).
\end{example}
Automatic differentiation is a formalization of last Example. Let
$x_1,\ldots,x_d$ be the input variables to the function,
$x_{d+1},\ldots,x_{D-1}$ be the intermediate variables, and $x_D$ the output
variable. Then the computation graph can be expressed as follows:
\begin{equation}\label{eq:forward_pass}
    \text{For }i=d+1,\ldots,D:\quad x_i=g_i(x_{Pa}(x_i))
\end{equation}
where the $g_i(\cdot)$ are elementary functions and $x_{Pa}(x_i)$ are the
parent nodes of the variable $x_i$ in the graph.

Recall that by definition $f=x_D$ and hence
$$\frac{\partial{f}}{\partial{x_D}}=1$$
For other variables $x_i$, we apply the chain rule 
\begin{equation}\label{eq:backward_pass}
    \frac{\partial{f}}{\partial{x_i}}=\displaystyle\sum_{x_j:x_i\in
    Pa(x_j)}\frac{\partial{f}}{\partial{x_j}}\frac{\partial{x_j}}{\partial{x_i}}=\displaystyle\sum_{x_j:x_i\in
    Pa(x_j)}\frac{\partial{f}}{\partial{x_j}}\frac{\partial{g_j}}{\partial{x_i}}
\end{equation}
where $Pa(x_j)$ is the set of parent nodes of $x_j$ in the computation graph.
Equation \ref{eq:forward_pass} is the \underline{forward pass}, whereas
\ref{eq:backward_pass} is the \underline{backward pass}.

The automatic differentiation approach works whenever we have a function that
can be expressed as a computation graph, where the elementary functions are
differentiable. 
\cleardoublepage
\section{Continuous Optimization}
Training a machine learning essentially involves identifying a good set of
parameters. What constitutes ``good'' is defined by the objective function.
Optimization algorithms are employed to locate the best possible value of this
function. Typically, the aim is to minimize the objective function, implying
that the best value is the minimum one.

\begin{figure}[!h]
   \centering
   \includegraphics[width=0.25\linewidth]{argmin_argmax}
   \caption{$\underset{x\in \mathbb{R}^n}\argmax\
   f(x)=\underset{x\in\mathbb{R}^n}\argmin\ {-f(x)}$}
\end{figure}

We will assume in this chapter that our objective function $f:\mathbb{R}^n\to
\mathbb{R}$ is differentiable, hence we have access to a gradient to help us
find the optimal value. Intuitively, finding the best value is like finding
the valleys of the objective function, and the gradients point us uphill. The
idea is to move downhill (opposite to the gradient) and hope to find the
deepest point.
\subsection{Conditions for the existence of the minimum} 
\begin{definition}
    $f$ is differentiable if the partial derivates
    $\frac{\partial{f}}{\partial{x_i}},i=1,\ldots,n$ exist and are continuous. 
\end{definition}
\begin{definition}
    $x^*\in \mathbb{R}^n$ is a (strict) \textbf{local minimum} of $f$ if there
    exists $\epsilon>0$ such that:
    $$f(x^*)(<)\leq f(x)\quad \forall x: \lVert x-x^*\rVert<\epsilon$$
\end{definition}
\begin{definition}
    $x^*\in \mathbb{R}^n$ is a (strict) \textbf{global minimum} of $f$ if 
    $$f(x^*)(<)\leq f(x)\quad \forall x\in\mathbb{R}^N$$
\end{definition}
\begin{center}
    \includegraphics[width=0.75\linewidth]{local_global_minimum}
\end{center}
\begin{definition}[First order conditions]
    If $x^*$ is a minimum point of $f$, then $\nabla f(x^*)=0$. Furthermore,
    if $\nabla f(x^*)=0$ for $x^*\in \mathbb{R}^n$, then $x^*$ can be either a
    (local) minimum, a (local) maximum or a saddle point of $f(x)$.
\end{definition}
Consequently, we want to find a point $x^*\in \mathbb{R}^n$ such that $\nabla
f(x^*)=0$. Those points are stationary points for $f$.
\begin{center}
    \includegraphics[width=\linewidth]{min_max_saddle}
\end{center}
\begin{definition}[Second order conditions]
    if $f$ is twice differentiable and if $\nabla f(x^*)=0$ and
    $\nabla^2f(x^*)$ (the hessian of $f$) is positive definitive then $x^*$ is
    a strict local minimum for $f$.
\end{definition}
\subsection{Algorithm to compute the minimum}

\paragraph{Iterative methods.} Given an initial vector $x_0\in \mathbb{R}^n$,
it is possible to approximate a solution to a given optimization problem by
applying iterative methods. In particular, by using these methods, one can
compute $x_{k+1}$ as follows:
$$x_{k+1}=g(x_k)$$
until convergence. In this case $g$ is an arbitrary function. Using these
methods, $x_k\to x^*$ for $k\to\infty$, where $x^*$ is a stationary point.

\paragraph{Descent methods} are iterative methods in which one can compute
$x_{k+1}$ as follows:
$$x_{k+1}=x_k+\alpha_kp_k$$
where $p_k\in \mathbb{R}^n$ and $\alpha_k\in \mathbb{R}$.
\begin{definition}
    $p_k$ is called a \textbf{descent direction} for $f$ in $x$ if there
    exists $\alpha_k>0$ such that:
    $$f(x_k+\alpha_k p_k)<f(x_k)$$
    In this case:
    $$p_k^T\nabla f(x_k)<0\quad\text{if } p\neq0$$
    In other words, descent direction is a direction that along that line
    decreases the function.
\end{definition}
and $\alpha_k$ is a positive parameter called \textbf{step size} that measures
the step along the direction $p_k$.

The direction $p_k$ corresponds in the \textbf{gradient
descent method} to $-\nabla f(x_k)$, thus
$$x_{k+1}=x_k-\alpha_k\nabla f(x_k)$$
The selection of $\alpha_k$ is crucial task for ensuring convergence to the
minimum of a function. A step size that is too small can lead to excessively
slow convergence, potentially never reaching the minimum, while a step size
that is too large may cause bouncing around the minimum without ever
converging to it.

\begin{center}
    \includegraphics[width=\linewidth]{gradient_step_size}
\end{center}

If $\alpha_k$ is chosen with the \textbf{backtracking procedure} (Armijo rule) then the
algorithm converges to a stationary point of $f$. The idea is to start from
an initial value for $\alpha_k$, and then reducing it as
$\alpha_k=\phi\alpha_k$ with $\phi<1$ until the following condition is met:
$$f(x_k-\alpha_k\nabla f(x_k))\leq f(x_k)-\sigma\alpha_k \lVert \nabla
f(x_k)\rVert^2$$
where $\sigma$ is typically 0.25 and lies in the range $(0,0.5)$. The typical value
of $\phi$ is also 0.5.

Since we cannot run infinite iterations (there exists a truncation error), a
\textbf{stopping criteria} is needed:
\begin{itemize}
    \item We can use the property of $x^*$, where $\nabla f(x^*)=0$, to test
        whether the approximation $x_k$ is close to the solution within a
        specified tolerance. The criteria can be:
        \begin{itemize}
            \item \textbf{Absolute criterion}: $\lVert \nabla
                f(x_k)\rVert<\tau_A$
            \item \textbf{Relative criterion}: $\frac{\lVert \nabla
                f(x_k)\rVert}{\lVert \nabla f(x_0)\rVert}<\tau_R$
        \end{itemize}
    \item We can set a maximum number, $k^*$, of iterations.
    \item Additionally, as an heuristic, we stop when the norm of the gradient
        starts to flatten.
\end{itemize}
\begin{verbatim}
   input: f, x_0
   while stopping criteria holds:
        p_k = - grad(f(x_k))
        a_k = backtrack_procedure
        x_k = x_k + a_k * p_k
        k = k + 1
     output: x_k
\end{verbatim}
\paragraph{Initialization.} The initial point $x_0$ influences the
local minimum that is found. It is usually chosen randomly within the range of
$[-1,1]$ or $[0,1]$. Moreover, if $f(x)$ is convex, then every stationary
point is a global minimum of $f(x)$ and the choice of $x_0$ isn't important.
Otherwise when $f(x)$ isn't convex, we have to choose an initial point as
closest as possible to the \textit{right} stationary point.

\mbox{}

The algorithm introduced doesn't always work well because of the following
reasons:
\begin{itemize}
    \item If the objective function has flat areas or potholes on its loss
        surface, the procedure might be too slow or lead to a poor solution.
        Techniques such momentum, that inherit the rate of descent from
        previous steps, are often able to navigate through local potholes and flat
        regions. The idea is akin a stone rolling down a hill, gathering
        speed as it rolls down.
    \item If the components of the gradient have very different magnitudes,
        this causes problems for gradient-descent methods. 
    \item If the objective function has a steep region in its loss surface, the
        descent direction within this area changes quickly. This rapid change
        increase the likelihood of the divergence.
    \item The objective function may present non-differentiable points,
        which can create problems in calculating the gradient.
\end{itemize}

\subsection{Convex functions}

A convex set is a set where, for any two points within the set, the line
segment connecting these two points entirely lies within the set.

\begin{center}
    \includegraphics[width=0.65\linewidth]{convex_set}
\end{center}

\begin{definition}
    Let $f:\Omega\subset \mathbb{R}^n\to \mathbb{R}$, where $\Omega$ is a
    convex set. The function $f$ is (strictly) \textbf{convex} if, $\forall x,y\in\Omega$
    and $\forall\theta:0\leq\theta\leq1$, the following inequality holds:
    $$f(\theta x+(1-\theta)y) (<) \leq\theta f(x)+(1-\theta)f(y)$$
\end{definition}
In other words, the function of a point lying on the segment connecting $x$
and $y$ is below the segment connecting $f(x)$ and $f(y)$.
\begin{center}
    \includegraphics[width=0.5\linewidth]{convex_function}
\end{center}
\subsubsection{Quadratic functions}
An example of a strictly convex function is the quadratic function, which take the
following form:
$$f(x)=\frac{1}{2}x^TBx+c^Tx+(w)$$
with $b,c\in \mathbb{R}^n$ and $A\in \mathbb{R}^{n\times n}$ symmetric
positive definitive.

A typical quadratic function is the least square:
$$\lVert Ax-b\rVert^2$$
\begin{proof}
    \begin{equation*}
        \begin{aligned}
            \frac{1}{2}\lVert Ax-b\rVert^2 &=\frac{1}{2}(Ax-b)^T(Ax-b)\\ 
                                &=\frac{1}{2}(x^TA^T-b^T)(Ax-b)\\
                                &=\frac{1}{2}(x^TA^TAx-b^TAx-x^TA^Tb+b^Tb)\\ 
                                &=\frac{1}{2}x^T\underbrace{A^TA}_{\color{blue}B}x\underbrace{-b^TA}_{\color{blue}c^T}x+\frac{1}{2}b^Tb
        \end{aligned}
    \end{equation*}
    which is exactly the same as the above form.
\end{proof}
\subsubsection{Properties}
\begin{enumerate}
    \item If $f$ is convex, any point of local minimum is also a global
        minimum. 
    \item While a convex function can have multiple local minima, a strictly
        convex function has only \underline{one} local minimum, which is also
        a global minimum.
        \begin{center}
            \includegraphics[width=0.75\linewidth]{strictly_convex}
        \end{center}
    \item If $f$ is convex and \underline{differentiable} any stationary point
        is a global minimum for $f$.
\end{enumerate}
\subsection{Variants of gradient descent algorithm}
\begin{itemize}
    \item Gradient descent with \textbf{momentum} method improves the
        convergence of the gradient descent method by memorizing and
        utilizing, at each step, what happened in the previous iteration. In
        particular:
        $$x_{k+1}=x_k-\alpha_k\nabla f(x_k)+\beta\Delta x_k$$
        where $\beta\in[0,1]$ and $\Delta x_k=x_k-x_{k-1}$ is the update
        obtained at iteration $k$.
        This update smooths the gradient updates and, thus, reduces the
        oscillations. This approach is analogous to a heavy ball in motion,
        where the momentum term represents the ball's resistance to change
        directions.
        \begin{center}
            \includegraphics[width=0.75\linewidth]{gradient_momentum}
        \end{center}
    \item In machine learning, we typically train models by finding the
        optimal vector of parameters, denoted as $\theta$, that minimize a
        loss function $L(\theta)$. We can see this loss functions as the
        aggregate of individual losses $L_n$ incurred by each of the $N$ data
        points in the training set. Thus, the loss function is expressed as:
        $$L(\theta)=\sum_{n=1}^NL_n(\theta)$$
        The gradient of such loss function is then computed as follows:
        $$\nabla(\theta)=\sum_{n=1}^N\nabla L_n(\theta_k)$$
        In gradient descent, the optimization is performed by using
        the full training set and by updating the vectors of parameters
        according to:
        $$\theta_{k+1}=\theta_k-\alpha_k\sum_{n=1}^N(\nabla L_n(\theta_k))$$
        Evaluating the sum gradient may require expensive evaluations of the
        gradients from all individual functions $L_n$. If we consider the term
        $\sum_{n=1}^N(\nabla L_N(\theta_k))$, we can reduce the amount of the
        computation by taking a sum over a \underline{smaller set} of $L_n$.
        \begin{itemize}
            \item \textit{batch} $\to$ \underline{all} $L_n$ functions;
            \item \textit{mini-batch} $\to$ randomly choose a
                \underline{subset} of $L_n$
                functions. This can be a single $L_n$ or more. Large
                mini-batches lead to more stable convergence, but the
                calculations will be more expensive. Small mini-batches are
                quick to estimate, and the noise in gradient estimation might
                help to escape from some bad local optima.
        \end{itemize}
        The technique is used by \textbf{stochastic gradient descent}. It
        requires more iterations to converge, but with a small enough
        $\alpha_k$, it almost surely converges to a local minimum. Why use it?
        If there are constraints, such as memory limitations.

        Moreover, with stochastic gradient descent, we speak about \textbf{epoch} and not
        iterations. An epoch refers to the iterations necessary to ``see" all
        the data.
        $$\nabla L(\theta)=\nabla L_{n_1}(\theta)\quad i=1,\ n_1\in\{1\dots N\}$$
        $$\nabla L(\theta)=\nabla L_{n_2}(\theta)\quad i=1,\ n_2\in\{1\dots
        N\}\setminus\{n_1\}$$
        The mini-batches do not repeat themselves before the entire epoch has
        been completed.
\end{itemize}
\cleardoublepage
\section{Probability and statistics}
\end{document}
